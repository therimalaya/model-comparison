\documentclass[12pt, a4paper]{article}
\usepackage{amsmath}
\usepackage{graphicx}
\usepackage{subcaption}
\usepackage{enumitem}
\usepackage{mathpazo}
\usepackage{setspace}

\begin{document}

\doublespacing

\begin{table}[ht]
  \centering
  \begin{tabular}{rll}
    Number of training samples              & $n$      & 50                \\
    Number of predictor variables           & $p$      & 15 and 40         \\
    Population coefficient of determination & $\rho^2$ & 0.5 and 0.9       \\
    Position of relevant components         & $\mathcal{P}$
                                            & $\triangleright$ 1, 2 \;
                                                         $\triangleright$ 1,  3 \; \newline
                                                         $\triangleright$ 2,  3 and \;
                                                         $\triangleright$ 1,  2, 3 \\
    Decay factor of eigenvalues of X        & $\eta$   & 0.5 and 0.9
  \end{tabular}
  \caption{Parameters used for simulating calibration sets}
  \label{tab:parameters}
\end{table}

\pagebreak

\begin{figure}[!ht]
  \centering
  \includegraphics[width=\textwidth]{pdf/effect-plot.pdf}
  \caption{Third order interaction effects}
  \label{fig:effect-plot}
\end{figure}

\pagebreak

\begin{figure}[!ht]
  \centering
  \includegraphics[width = \textwidth]{pdf/prediction-error-15-1.pdf}
  \caption[Prediction Error]{Average Prediction Error for designs with 15 predictor
    variables where Coefficient of determination is 0.5}
  \label{fig:pred-error-15-1}
\end{figure}

\pagebreak

\begin{figure}[!ht]
  \centering
  \includegraphics[width = \textwidth]{pdf/prediction-error-15-2.pdf}
  \caption[Prediction Error]{Average Prediction Error for designs with 15 predictor
    variables where Coefficient of determination is 0.9}
  \label{fig:pred-error-15-2}
\end{figure}

\pagebreak

\begin{figure}[!hptb]
  \centering
  \includegraphics[width = \textwidth]{pdf/prediction-error-40-1.pdf}
  \caption[Prediction Error]{Average Prediction Error for designs with 40 predictor
    variables where Coefficient of determination is 0.5}
  \label{fig:pred-error-40-1}
\end{figure}

\pagebreak

\begin{figure}[!hptb]
  \centering
  \includegraphics[width = \textwidth]{pdf/prediction-error-40-2.pdf}
  \caption[Prediction Error]{Average Prediction Error for designs with 40 predictor
    variables where Coefficient of determination is 0.9}
  \label{fig:pred-error-40-2}
\end{figure}

\pagebreak

\begin{figure}[!ht]
  \centering
  \includegraphics[width=\textwidth]{pdf/est-combined-plot.pdf}
  \caption{Correlation between true and estimated beta coefficient and Beta Estimation Error}
  \label{fig:est-error-combined}
\end{figure}


\end{document}

